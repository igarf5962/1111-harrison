\documentclass[12pt]{article}
\usepackage[margin=0.75in]{geometry}
\usepackage{graphicx}
\setlength{\parindent}{0mm}

\begin{document}

{\centering
\large University of North Georgia \par
\large College of Science and Mathematics \par
\large Department of Physics \par
\large PHYS 1111 - Introductory Physics I - Summer 2018 \par
}
\hfill \break \vspace{-4mm}

``The noblest pleasure is the joy of understanding.'' -Leonardo da Vinci
\hfill \break

\underline{\textbf{General Information}} \par
Instructor: Dr. Nathan Harrison \par
Office: Science 113 \par
Email: Standard UNG email address \par
Office Hours: See the pdf file of my schedule.
\hfill \break

\underline{\textbf{Required Materials}} \par
Textbook: College Physics 11th Edition w/ Webassign by Serway \par
Scientific or graphing calculator, ruler, protractor, paper, scan-tron forms \par
GitHub account \par
SageMath (Cloud account or your own installation) \par
Java JDK 1.8 or greater
\hfill \break

\underline{\textbf{Course Description}} \par
This is an introductory course which will include material from mechanics, thermodynamics, and waves;
and is the first course in a two-semester Physics sequence.
Elementary algebra and trigonometry will be used.
The prerequisite is MATH 1113 or MATH 1450.
The lecture is three credits while the lab is one credit. (see the lab syllabus for more details)
\hfill \break

\underline{\textbf{Course Description}} \par
We will be covering Parts 1-3 (Mechanics, Thermodynamics, and Vibrations and Waves) of the textbook which consists of 14 chapters.
We will cover approximately 2-3 chapters per week and one corresponding lab per chapter (see the lab syllabus for more details). A detailed schedule can be found at the end of this syllabus.
\hfill \break

\end{document}
